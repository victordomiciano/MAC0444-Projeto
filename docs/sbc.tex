\documentclass{article}

% biber

\usepackage[brazilian]{babel}
\usepackage[utf8]{inputenc}
\usepackage{graphicx}
\usepackage{mathtools}
\usepackage{amsthm}
\usepackage{thmtools,thm-restate}
\usepackage{amsfonts}
\usepackage{amssymb}
\usepackage{hyperref}
\usepackage{enumitem}
\usepackage[justification=centering,singlelinecheck=false]{caption}
\usepackage{indentfirst}
\usepackage{listings}
\usepackage[x11names, rgb]{xcolor}
\usepackage{tikz}
\usepackage{hyperref}
\usepackage{subcaption}
\usepackage{booktabs}
\usepackage{linegoal}
\usepackage{csquotes}
\usetikzlibrary{snakes,arrows,shapes}

\DeclareMathOperator*{\argmin}{arg\,min}
\DeclareMathOperator*{\argmax}{arg\,max}
\DeclareMathOperator*{\Val}{\text{Val}}
\DeclareMathOperator*{\Ch}{\text{Ch}}
\DeclareMathOperator*{\Pa}{\text{Pa}}
\DeclareMathOperator*{\Sc}{\text{Sc}}
\newcommand{\ov}{\overline}
\newcommand{\region}{\mathcal}

\newcommand\defeq{\mathrel{\overset{\makebox[0pt]{\mbox{\normalfont\tiny\sffamily def}}}{=}}}

\captionsetup[table]{labelsep=space,labelfont=bf}

\setlistdepth{9}
\newlist{deepitemize}{itemize}{9}
\setlist[deepitemize,1]{label=$\bullet$}
\setlist[deepitemize,2]{label=$\bullet$}
\setlist[deepitemize,3]{label=$\bullet$}
\setlist[deepitemize,4]{label=$\bullet$}
\setlist[deepitemize,5]{label=$\bullet$}
\setlist[deepitemize,6]{label=$\bullet$}
\setlist[deepitemize,7]{label=$\bullet$}
\setlist[deepitemize,8]{label=$\bullet$}
\setlist[deepitemize,9]{label=$\bullet$}

\newcommand{\set}[1]{\mathbf{#1}}
\newcommand{\pr}{\mathbb{P}}
\newcommand{\eps}{\varepsilon}
\renewcommand{\implies}{\Rightarrow}

\newcommand{\bigo}{\mathcal{O}}

\newtheorem{assumption}{Suposição}

\setlength{\parskip}{1em}

\lstset{frameround=fttt,
	numbers=left,
	breaklines=true,
	keywordstyle=\bfseries,
	basicstyle=\ttfamily,
}

\newcommand{\code}[1]{\lstinline[mathescape=true]{#1}}
\newcommand{\mcode}[1]{\lstinline[mathescape]!#1!}
\newcommand{\dset}[1]{\mathcal{#1}}

\title{\Huge Projeto\\~\\\LARGE Sistemas Baseados em Conhecimento (MAC0444)}
\date{}
\author{\Large{Renato Lui Geh}\\\large{NUSP\@: 8536030}\\}

\begin{document}

\maketitle
\newpage

\section{Suposições}

Neste projeto tivemos de assumir certas suposições com relação ao \textit{dataset}. Para construir
o OWL, usamos o arcabouço \code{imdbpy}.

\subsection{Filmes}

Ao analisar os dados do dataset do IMDb, notamos que haviam vários títulos que tinham uma definição
diferente de filme. Séries e miniséries de TV são consideradas pelo IMDb como filmes. Similarmente
\textit{shows} de televisão, como a apresentação dos prêmios Oscar e Emmy também entram na mesma
categoria de filmes. Apesar do IMDb considerar diferentes tipos de filmes, como \code{tv movie} e
\code{movie}, essa diferenciação ainda não é uma boa classificação, já que haviam exceções a regra.

Para definirmos de forma mais rigorosa o que é um filme, tivemos que supor alguns conceitos sobre
filmes. Em particular, consideramos um filme um título do IMDb que segue as seguintes regras:

\begin{enumerate}
  \item Não tem ``episódios'';
  \item Tem um conjunto de ``diretores'';
  \item Tem ano de lançamento;
  \item Tem um conjunto de ``atores'';
  \item É do tipo \code{movie}.
\end{enumerate}

Com estas fortes retrições, foi possível selecionar um conjunto consistente de filmes.

Na ontologia, chamamos de \code{Movie} o conceito de filme. Definimos um \code{Movie} como uma
subclasse de \code{Project}, conceito da ontologia \code{FOAF-modified} dada. Todo \code{Movie}
possui as propriedades de dado \code{movieTitle} do tipo \code{xsd:string} e \code{releaseYear} do
tipo \code{xsd:positiveInteger}. Estas propriedades representam, respectivamente, o título do
filme e o ano de lançamento.

Chamaremos de \code{projeto} o prefixo da ontologia que criamos, e \code{foaf-modified} a ontologia
\code{FOAF-modified} dada no enunciado. A relação (i.e.\ propriedade de objeto)
\code{foaf-modified:maker} de um \code{foaf-modified:Project} foi usado para denotar todos os
colaboradores (i.e.\ atores e diretores) de um \code{projeto:Movie}.

\subsection{Atores e diretores}

Para os conceitos \code{Actor} e \code{Director}, usamos o conceito \code{foaf-modified:Person}
como superclasse para ambos. As propriedades de dado de ambas são \code{familyName},
\code{firstName} e \code{gender}, todas da ontologia \code{foaf-modified}. Estas representam,
respectivamente, o sobrenome, primeiro nome e gênero de um \code{Person}. Para \code{gender},
estabelecemos que os possíveis valores poderiam ser ``\code{male}'', ``\code{female}'' ou
``\code{nil}''.

Ao criar o \code{.owl} a partir do dataset, fizemos as seguintes suposições sobre as pessoas:

\begin{enumerate}
  \item O primeiro nome de uma pessoa no IMDb é o seu \code{firstName};
  \item O último nome de uma pessoa no IMDb é o seu \code{familyName};
  \item O gênero de uma pessoa foi definido a partir de consultas no \code{imdbpy}.
\end{enumerate}

Com relação a suposição 3, o dataset do IMDb não especificam gênero de uma pessoa. No entanto, o
\code{imdbpy} gera uma tabela em SQL com os gêneros de cada ator, tendo estes valores \code{'f'},
\code{'m'} ou um valor nulo quando não especificado. Escolhemos seguir a mesma convenção neste
projeto.

A relação \code{foaf-modified:made} mapeia um \code{Person} em um \code{Movie}. Seu inverso é
\code{foaf-modified:maker}. Criamos duas relações subclasse de \code{made}: \code{actsIn} e
\code{directs}, mapeando, respectivamente, um ator em um filme que ele atua e um diretor em um
filme que ele dirige.

Para selecionarmos a lista de pessoas iniciais (i.e.\ Uma Thurman, Harvey Keitel, etc.), assumimos
que a pessoa de interesse é o primeiro resultado da busca de seu nome concatenado com
``\code{(I)}''. Isso garante que o homônimo escolhido é sempre o primeiro em popularidade no IMDb.

\subsection{Relações}

Resumimos as relações existentes na tabela abaixo:
\newpage

\begin{table}[h]
  \centering
  \begin{tabular}{lllc}
    Nome & Domínio & Imagem & Superclasse\\
    \midrule
    \code{made} & \code{Person} & \code{Movie} & -\\
    \code{maker} & \code{Movie} & \code{Person} & -\\
    \code{actsIn} & \code{Actor} & \code{Movie} & \code{made}\\
    \code{directs} & \code{Director} & \code{Movie} & \code{made}\\
  \end{tabular}
  \captionsetup{justification=raggedright}
  \caption{Relações entre conceitos. As colunas indicam o nome da relação, o domínio e imagem, e a
  superclasse da relação. Como \code{made} e \code{maker} têm como superclasse a raíz comum de
  todas as relações, omitimos e no lugar indicamos com o caractere '-'.}
\end{table}

\subsection{Instâncias}

Foram escolhidas as instâncias de \code{Movie}, \code{Actor} e \code{Director} como descrito no
enunciado. Depois de aplicadas as restrições descritas nas subseções 1.1 e 1.2, geramos o
\code{.owl} com a sintaxe Manchester.

Cada instância de \code{Movie} é do formato:

\begin{lstlisting}[frame=single,mathescape=true]
Individual: projeto:movieTitle

  Types:
    projeto:Movie

  Facts:
    projeto:movieTitle "Movie Title",
    foaf-modified:maker projeto:director1,
    $\vdots$
    foaf-modified:maker projeto:directorN,
    foaf-modified:maker projeto:actor1,
    $\vdots$
    foaf-modified:maker projeto:actorM,
    projeto:releaseYear Y
\end{lstlisting}

Onde \code{Y} é um inteiro positivo. A lista de atores e diretores não são ordenadas como no
exemplo. Para uma instância de \code{Actor} ou \code{Director} temos o seguinte formato:

\newpage

\begin{lstlisting}[frame=single,mathescape=true]
Individual: projeto:personName

  Types:
    projeto:Actor,
    projeto:Director

  Facts:
    projeto:actsIn projeto:movie1,
    $\vdots$
    projeto:actsIn projeto:movieN,
    projeto:directs projeto:movie1,
    $\vdots$
    projeto:directs projeto:movieM,
    foaf-modified:familyName "Name",
    foaf-modified:firstName "Person",
    foaf-modified:gender ["male" ou "female" ou "nil"]
\end{lstlisting}

Para todas as instâncias, o nome da instância é baseado no seu título ou nome. Usamos o formato
conhecido como \code{camelCase} para formatar os nomes das instâncias. Para atores que não são
diretores, omitimos a vírgula anterior e \code{projeto:Diretor} em \code{Types:}. Fazemos o análogo
para diretores que não atuam em nenhum filme.

\subsection{Unicidade}

Quando definimos os nomes dos atores e diretores, consideramos que se alguma pessoa $p$ possui
mesmo \code{firstName} e \code{familyName} que outra pessoa $q$, então $p=q$. Isso simplifica a
ontologia, tratando homônimos como a mesma pessoa. Também supomos que todo filme que títulos
diferentes. Apesar de não ser verdade, esta suposição simplifica bastante a ontologia.

Uma consequência das suposições anteriores é que não precisamos usar a propriedade
\code{DifferentFrom} do OWL, evitando que o arquivo cresça exponencialmente, já que toda junção de
primeiro com último nome será diferente, assim como todo título de filme será distinto.

\section{Consultas}


Neste projeto tivemos que realizar consultas em SPARQL para responder as seguintes questões:

\begin{enumerate}
  \item Quais atores participaram do filme F?
  \item Quais filmes foram dirigidos pelo diretor D?
  \item Em quais filmes o ator X atuou?
  \item Em quais filmes o ator X atuou junto com Y?
  \item Quem foram os diretores dos filmes nos quais os atores X e Y atuam juntos?
  \item Qual o diretor que mais dirigiu filmes do ator X?
  \item Qual o ator que mais aparece nos filmes do diretor D?
  \item Entre os anos N1 e N2, quais diretores dirigiram filmes onde X e Y aparecem?
  \item Entre os anos N1 e N2, quais atores atuaram juntos nos filmes onde X e Y aparecem?
  \item Quais filmes do diretor do filme F possuem X ou Y como atores?
\end{enumerate}

Obs.: Para estas consultas, introduzimos \code{FILTER}s para selecionar os resultados desejados, de forma que o filme (\code{Movie}) é introduzido pelo seu título \code{projeto:movieTitle} no filtro em questão. Para introduzir uma pessoa (\code{Actor} ou \code{Director}) utilizamos dois filtros combinados, um para seu primeiro nome \code{foaf-modified:firstName} e outro para seu segundo nome \code{foaf-modified:familyName}.
\subsection{Consulta 1}

Quais atores participaram do filme F \code{(?f = "Isle of Dogs")?}

\begin{lstlisting}[basicstyle=\ttfamily,frame=single]
SELECT ?1x ?2x
WHERE {
  ?x projeto:actsIn ?m .
  ?m projeto:movieTitle ?f .
  ?x foaf-modified:firstName ?1x .
  ?x foaf-modified:familyName ?2x .
  FILTER REGEX(?f, "(^Isle of Dogs$)") .
}
\end{lstlisting}
\subsection{Consulta 2}
Quais filmes foram dirigidos pelo diretor D (\code{?1d = "Wes"} \code{?2d = "Anderson"})?
\begin{lstlisting}[basicstyle=\ttfamily,frame=single]
SELECT ?f
WHERE {
  ?d projeto:directs ?m .
  ?m projeto:movieTitle ?f .
  ?d foaf-modified:firstName ?1d .
  ?d foaf-modified:familyName ?2d .
  FILTER REGEX(?1d, "(^Wes?)") .
  FILTER REGEX(?2d, "(^Anderson?)") .
}
\end{lstlisting}
\subsection{Consulta 3}
Em quais filmes o ator X (\code{?1x = "Samuel"} \code{?2x = "Jackson"}) atuou?
\begin{lstlisting}[basicstyle=\ttfamily,frame=single]
SELECT ?f
WHERE {
  ?x projeto:actsIn ?m .
  ?m projeto:movieTitle ?f .
  ?x foaf-modified:firstName ?1x .
  ?x foaf-modified:familyName ?2x .
  FILTER REGEX(?1x, "(^Samuel?)") .
  FILTER REGEX(?2x, "(^Jackson?)") .
}
\end{lstlisting}
\subsection{Consulta 4}
Em quais filmes o ator X (\code{?1x = "Uma"} \code{?2x = "Thurman"}) atuou junto com Y (\code{?1y = "Samuel"} \code{?2y = "Jackson"})?
\begin{lstlisting}[basicstyle=\ttfamily,frame=single]
SELECT DISTINCT ?f
WHERE {
  ?x projeto:actsIn ?m .
  ?x foaf-modified:firstName ?1x .
  ?x foaf-modified:familyName ?2x .
  FILTER REGEX(?1x, "(^Uma?)") .
  FILTER REGEX(?2x, "(^Thurman?)") .
  ?y projeto:actsIn ?m .
  ?m projeto:movieTitle ?f .
  ?y foaf-modified:firstName ?1y .
  ?y foaf-modified:familyName ?2y .
  FILTER REGEX(?1y, "(^Samuel?)") .
  FILTER REGEX(?2y, "(^Jackson?)") .
}
\end{lstlisting}
\subsection{Consulta 5}
Quem foram os diretores dos filmes nos quais os atores X (\code{?1x = "Uma"} \code{?2x = "Thurman"}) e Y (\code{?1y = "Samuel"} \code{?2y = "Jackson"}) atuam juntos?
\begin{lstlisting}[basicstyle=\ttfamily,frame=single]
SELECT DISTINCT ?1d ?2d
WHERE {
  ?x projeto:actsIn ?m .
  ?x foaf-modified:firstName ?1x .
  ?x foaf-modified:familyName ?2x .
  FILTER REGEX(?1x, "(^Uma?)") .
  FILTER REGEX(?2x, "(^Thurman?)") .
  ?y projeto:actsIn ?m .
  ?y foaf-modified:firstName ?1y .
  ?y foaf-modified:familyName ?2y .
  FILTER REGEX(?1y, "(^Samuel?)") .
  FILTER REGEX(?2y, "(^Jackson?)") .
  ?d projeto:directs ?m .
  ?d foaf-modified:firstName ?1d .
  ?d foaf-modified:familyName ?2d .
}
\end{lstlisting}
\subsection{Consulta 6}
Qual o diretor que mais dirigiu filmes do ator X (\code{?1x = "Frances"} \code{?2x = "Mcdormand"})?
\begin{lstlisting}[basicstyle=\ttfamily,frame=single]
SELECT ?1d ?2d WHERE {
  {
    SELECT ?d (COUNT(?d) AS ?count)
    WHERE {
      ?x projeto:actsIn ?m .
      ?x foaf-modified:firstName ?1x .
      ?x foaf-modified:familyName ?2x .
      FILTER REGEX(?1x, "(^Frances?)") .
      FILTER REGEX(?2x, "(^Mcdormand?)") .
      ?d projeto:directs ?m .
    } GROUP BY ?d
  }
  {
    SELECT (MAX(?count) AS ?max)
    WHERE {
      {
        SELECT ?d (COUNT(?d) AS ?count)
        WHERE {
          ?x projeto:actsIn ?m .
          ?x foaf-modified:firstName ?1x .
          ?x foaf-modified:familyName ?2x .
          FILTER REGEX(?1x, "(^Frances?)") .
          FILTER REGEX(?2x, "(^Mcdormand?)") .
          ?d projeto:directs ?m .
        } GROUP BY ?d
      }
    } 
  }
  FILTER (?count = ?max)
  ?d foaf-modified:firstName ?1d .
  ?d foaf-modified:familyName ?2d .
}
\end{lstlisting}
\subsection{Consulta 7}
Qual o ator que mais aparece nos filmes do diretor D (\code{?1d = "Wes"} \code{?2d = "Anderson"})?
\begin{lstlisting}[basicstyle=\ttfamily,frame=single]
SELECT ?1x ?2x WHERE {
  {
    SELECT ?x (COUNT(?x) AS ?count)
    WHERE {
      ?d projeto:directs ?m .
      ?d foaf-modified:firstName ?1d .
      ?d foaf-modified:familyName ?2d .
      FILTER REGEX(?1d, "(^Wes?)") .
      FILTER REGEX(?2d, "(^Anderson?)") .
      ?x projeto:actsIn ?m .
    } GROUP BY ?x
  }
  {
    SELECT (MAX(?count) AS ?max)
    WHERE {
      {
        SELECT ?x (COUNT(?x) AS ?count)
        WHERE {
          ?d projeto:directs ?m .
          ?d foaf-modified:firstName ?1d .
          ?d foaf-modified:familyName ?2d .
          FILTER REGEX(?1d, "(^Wes?)") .
          FILTER REGEX(?2d, "(^Anderson?)") .
          ?x projeto:actsIn ?m .
        } GROUP BY ?x
      }
    }
  }
  FILTER (?count = ?max) .
  ?x foaf-modified:firstName ?1x .
  ?x foaf-modified:familyName ?2x .
}
\end{lstlisting}
\subsection{Consulta 8}
Entre os anos N1 e N2, quais diretores dirigiram filmes onde X (\code{?1x = "Uma"} \code{?2x = "Thurman"}) e Y (\code{?1y = "Samuel"} \code{?2y = "Jackson"}) aparecem?
\begin{lstlisting}[basicstyle=\ttfamily,frame=single]
SELECT DISTINCT ?1d ?2d
WHERE {
  ?x projeto:actsIn ?m .
  
  ?m projeto:releaseYear ?year .
  FILTER (?year >= 1980 && ?year <= 2001) .
  ?x foaf-modified:firstName ?1x .
  ?x foaf-modified:familyName ?2x .
  FILTER REGEX(?1x, "(^Uma?)") .
  FILTER REGEX(?2x, "(^Thurman?)") .
  ?y projeto:actsIn ?m .
  ?y foaf-modified:firstName ?1y .
  ?y foaf-modified:familyName ?2y .
  FILTER REGEX(?1y, "(^Samuel?)") .
  FILTER REGEX(?2y, "(^Jackson?)") .
  ?d projeto:directs ?m .
  ?d foaf-modified:firstName ?1d .
  ?d foaf-modified:familyName ?2d .
}
\end{lstlisting}
\subsection{Consulta 9}
Entre os anos N1 e N2, quais atores atuaram juntos nos filmes onde X (\code{?1x = "Bill"} \code{?2x = "Murray"}) e Y (\code{?1y = "Frank"} \code{?2y = "Pellegrino"}) aparecem?
\begin{lstlisting}[basicstyle=\ttfamily,frame=single]
SELECT DISTINCT ?1a ?2a ?1b ?2b
WHERE {
  {
    SELECT ?m
    WHERE {
      ?x projeto:actsIn ?m .
      
      ?m projeto:releaseYear ?year .
      FILTER (?year >= 1980 && ?year <= 2004) .
      ?x foaf-modified:firstName ?1x .
      ?x foaf-modified:familyName ?2x .
      FILTER REGEX(?1x, "(^Bill?)") .
      FILTER REGEX(?2x, "(^Murray?)") .
      ?y projeto:actsIn ?m .
      ?y foaf-modified:firstName ?1y .
      ?y foaf-modified:familyName ?2y .
      FILTER REGEX(?1y, "(^Frank?)") .
      FILTER REGEX(?2y, "(^Pellegrino?)") .
    }
  }
  ?a projeto:actsIn ?m .
  ?b projeto:actsIn ?m .
  FILTER (?a != ?b) .
  ?a foaf-modified:firstName ?1a .
  ?a foaf-modified:familyName ?2a . 
  ?b foaf-modified:firstName ?1b .
  ?b foaf-modified:familyName ?2b .
} ORDER BY ?1a ?2a ?1b ?2b
\end{lstlisting}
\subsection{Consulta 10}
Quais filmes do diretor do filme F \code{?f = "Pulp Fiction"} possuem X (\code{?1x = "Clive"} \code{?2x = "Owen"}) ou Y (\code{?1y = "Jamie"} \code{?2y = "Dunno"}) como atores?
\begin{lstlisting}[basicstyle=\ttfamily,frame=single]
SELECT DISTINCT ?f
WHERE {
  {
    SELECT ?d
    WHERE {
      ?m projeto:movieTitle ?f .
      FILTER REGEX(?f, "(^Pulp Fiction?)") .
      ?d projeto:directs ?m .
    }
  }
  ?d projeto:directs ?m .
  ?a projeto:actsIn ?m .
  ?a foaf-modified:firstName ?1a .
  ?a foaf-modified:familyName ?2a .
   
  ?m projeto:movieTitle ?f .
  FILTER ((REGEX(?1a, "(^Clive$)") && REGEX(?2a, "(^Owen$)")) || (REGEX(?1a, "(^Jamie$)") && REGEX(?2a, "(^Dunno$)"))) .
}
\end{lstlisting}
\section{Resultados}

A seguir estão as tabelas com os resultados de cada consulta realizada anteriormente.

\begin{table}[htbp]
\begin{center}
\begin{tabular}{ll}

\textbf{1x} & \textbf{2x} \\ 
Harvey & Keitel \\ 
Kunichi & Nomura \\ 
Jeff & Goldblum \\ 
Akira & Ito \\ 
Frances & Mcdormand \\ 
Akira & Takayama \\ 
Edward & Norton \\ 
Liev & Schreiber \\ 
Frank & Wood \\ 
F & Abraham \\ 
Bryan & Cranston \\ 
Yoko & Ono \\ 
Fisher & Stevens \\ 
Tilda & Swinton \\ 
Koyu & Rankin \\ 
Greta & Gerwig \\ 
Yojiro & Noda \\ 
Courtney & Vance \\ 
Mari & Natsuki \\ 
Kara & Hayward \\ 
Scarlett & Johansson \\ 
Bill & Murray \\ 
Bob & Balaban \\ 
\end{tabular}
\end{center}
\caption{Resultado da Consulta 1}
\label{}
\end{table}

\begin{table}[htbp]
\begin{center}
\begin{tabular}{l}

\textbf{f} \\ 
Hotel Chevalier \\ 
Rushmore \\ 
Moonrise Kingdom Animated Book Short \\ 
The Life Aquatic with Steve Zissou \\ 
Bottle Rocket \\ 
Cousin Ben Troop Screening with Jason Schwartzman \\ 
Castello Cavalcanti \\ 
The Grand Budapest Hotel \\ 
The Royal Tenenbaums \\ 
Isle of Dogs \\ 
Prada Candy \\ 
The Darjeeling Limited \\ 
Come Together A Fashion Picture in Motion \\ 
Fantastic Mr Fox \\ 
Moonrise Kingdom \\ 
\end{tabular}
\end{center}
\caption{Resultado da Consulta 2}
\label{}
\end{table}

\begin{table}[htbp]
\begin{center}
\begin{tabular}{l}

\textbf{f} \\ 
Inglourious Basterds \\ 
Jackie Brown \\ 
Django Unchained \\ 
Youre Still Not Fooling Anybody \\ 
Kill Bill Vol 2 \\ 
Boffo Tinseltowns Bombs and Blockbusters \\ 
Pulp Fiction \\ 
Off the Menu The Last Days of Chasens \\ 
The Hateful Eight \\ 
\end{tabular}
\end{center}
\caption{Resultado da Consulta 3}
\label{}
\end{table}

\begin{table}[htbp]
\begin{center}
\begin{tabular}{l}

\textbf{f} \\ 
Youre Still Not Fooling Anybody \\ 
Kill Bill Vol 2 \\ 
Boffo Tinseltowns Bombs and Blockbusters \\ 
Pulp Fiction \\ 
\end{tabular}
\end{center}
\caption{Resultado da Consulta 4}
\label{}
\end{table}

\begin{table}[htbp]
\begin{center}
\begin{tabular}{ll}

\textbf{1d} & \textbf{2d} \\ 
Bill & Couturie \\ 
Mike & White \\ 
Quentin & Tarantino \\ 
\end{tabular}
\end{center}
\caption{Resultado da Consulta 5}
\label{}
\end{table}

\begin{table}[htbp]
\begin{center}
\begin{tabular}{ll}

\textbf{1d} & \textbf{2d} \\ 
Joel & Coen \\ 
Ethan & Coen   \\ 
\end{tabular}
\end{center}
\caption{Resultado da Consulta 6}
\label{}
\end{table}

\begin{table}[htbp]
\begin{center}
\begin{tabular}{ll}

\textbf{1x} & \textbf{2x} \\ 
Jason & Schwartzman \\ 
Bill & Murray  \\ 
\end{tabular}
\end{center}
\caption{Resultado da Consulta 7}
\label{}
\end{table}

\begin{table}[htbp]
\begin{center}
\begin{tabular}{ll}

\textbf{1d} & \textbf{2d} \\ 
Mike & White \\ 
Quentin & Tarantino  \\ 
\end{tabular}
\end{center}
\caption{Resultado da Consulta 8}
\label{}
\end{table}

\begin{table}[htbp]
\begin{center}
\begin{tabular}{llll}

\textbf{1a} & \textbf{2a} & \textbf{1b} & \textbf{2b} \\ 
Alec & Baldwin & Bill & Murray \\ 
Alec & Baldwin & Edward & Saxon \\ 
Alec & Baldwin & Frank & Pellegrino \\ 
Bill & Murray & Alec & Baldwin \\ 
Bill & Murray & Edward & Saxon \\ 
Bill & Murray & Frank & Pellegrino \\ 
Edward & Saxon & Alec & Baldwin \\ 
Edward & Saxon & Bill & Murray \\ 
Edward & Saxon & Frank & Pellegrino \\ 
Frank & Pellegrino & Alec & Baldwin \\ 
Frank & Pellegrino & Bill & Murray \\ 
Frank & Pellegrino & Edward & Saxon \\ 
\end{tabular}
\end{center}
\caption{Resultado da Consulta 9}
\label{}
\end{table}

\begin{table}[htbp]
\begin{center}
\begin{tabular}{l}

\textbf{f} \\ 
Grindhouse \\ 
Death Proof \\ 
Sin City \\ 
\end{tabular}
\end{center}
\caption{Resultado da Consulta 10}
\label{}
\end{table}

%--------------------------------------------------------------------------------------------------
\end{document}
